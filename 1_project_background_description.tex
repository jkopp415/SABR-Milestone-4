\section*{Project Background Description}

Rotating Detonation Engines (RDEs) are a novel method of combusting and accelerating gas for rocket propulsion. The key difference between RDEs and conventional engines is that the flame front moves beyond the local speed of sound in a mechanism referred to as detonation, a form of “Pressure-Gain Combustion” (PGC). This method of combusting propellants allows for much higher specific impulse, which in turn allows smaller engines and lower mass-flows to produce equivalent performance compared to a conventional (constant pressure) engine. In the context of this project, the overarching goal is to design, manufacture, and test-fire an architecture capable of using atmospheric air as an oxidizer and design with features that seek to integrate the system into a flight vehicle. RDEs are far more compact engines than conventional methods and have been shown through extensive research to have significant potential as power plants in hypersonic vehicles, guided munitions, and other compact, high-speed vehicles. In its current standing, the project is evaluating the minimum combustor size required to achieve stable detonation based on the propellants selected (Gaseous Hydrogen and Air at 79\% Nitrogen and 21\% Oxygen).