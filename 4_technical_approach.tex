\section*{Technical Approach}

System performance is estimated by two group-generated software tools that predict the performance of the feed system and combustor by outputting critical dimensions and information about the working fluids. This information is translated into CAD via SolidWorks, where our design follows proper DFM practices to ensure the parts produced are cheap and easy to manufacture. Additional performance metrics such as material stress, thermal performance, and flow coefficients can be provided by simulation software such as Ansys mechanical and fluent to drive design choices but will require validation. After manufacturing the test stand, LabVIEW and Synnax will be utilized as the team’s supervisory control and data acquisition (SCADA) system to collect data as tests are executed precisely. The testing campaign involves starting with a stoichiometric Air and Hydrogen mixture to validate the detonation capability. Once complete, the engine will sweep through a range of equivalence ratio and mass flow/flux rate conditions to develop an operational map and validate the capabilities of our fuel/oxidizer combination. The data produced by all these tests will be post-processed by PERL’s “Black Pearl” post-processing algorithms, allowing for the important parameters to be distilled from each run.